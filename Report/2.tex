\chapter{Let's get more confident with our little friend op-amp}
We designed a non-inverting amplifier with a variable gain by using a trimmer. The second circuit designed was a summing amplifier with an unitary gain. We built a current source generator of $1$ mA and tested it with a variable load. We tested the efficacy of the emitter follower configuration in mismatching the source's impedence. Last we designed a differential amplifier with a predetermined gain.
\section{Materials}
\begin{itemize}
\item Operational amplifier uA741
\item Resistors and trimmers
\item Power supply RIGOL DP831A
\item Waveform generator RIGOL DG1032
\item Multimeter RIGOL DM3068
\item Oscilloscope RIGOL MS02102A
\item Two capacitance of nominal value of $100$nF
\end{itemize}
\section{Experiment setup}
In each circuit we powered the op-amp with a $\pm15$ V DC voltage and, in order to reduce possible noises, we added two 100nF capacitors connecting the op-amp's pins for the power supply with the ground. The input signal has a frequency of 100 Hz and a peak-peak voltage of 1V except for the differential amplifier.For every specific circuit we designed them as follow:
\begin{itemize}
\item Inverting amplifier: we placed a 10k$\Omega$ trimmer along the feedback branch in series to a resistor $R_f = 983.9 \pm 0.1 \Omega$. In order to have a minimal gain of 5, we used $R_{in} = 199.84\pm 0.03 \Omega$ as in figure \eqref{Non-inverting variable amplifier}.
\item Summing amplifier: caring for the simplest calculations, we used $R_1 = 1484.7 \pm 0.2 \Omega \simeq R_2= 1483.5\pm 0.2\Omega$ so the equation is $\displaystyle\frac{v_1+v_2}{2}\left(1+\frac{R_4}{R_3}\right)$. For obtaining the sum of the input in output, we had to choose $R_3 = R_4 = 1001.3 \pm 0.1 \Omega$. The inputs $v_1$ and $v_2$ are the same 100 Hz, 1 V peak-peak sine wave signal.

\begin{figure}[H]
\centering
\begin{minipage}{.5\textwidth}
  \centering
\begin{circuitikz}
\draw(0,0) node[op amp] (opamp) {}
	%(opamp.+) node[left] {$v_+$}
	(opamp.+) ++ (-.3,0) node[ground] {} -- (opamp.+) 
	(opamp.out) to[short] (1.8,0) node[right] {$v_o$}
	(opamp.down) ++(0,-.7) node[below] {$-v_{cc}$} -- (opamp.down)
	(opamp.up) ++ (0,.7) node[above] {$+v_{cc}$} -- (opamp.up)
	(opamp.down) ++ (0,-.25)to[C,/tikz/circuitikz/bipoles/length=1cm] (1,-.8)node[ground,rotate = 90,yshift = 1em] {}
	(opamp.up) ++ (0,.25)to[C,/tikz/circuitikz/bipoles/length=1cm] (1,.8)node[ground,rotate = 90,yshift = 1em] {};
	\draw(-4,-1) to[sV,l=$v_{in}$] (-4,.5) to[R=$R_{in}$] (-2,.5) to[short] (opamp.-);
	\draw(-4,-1) node[ground] {};
	
	\draw(-1.5,.5) to[short](-1.5,2.2) to[R=$R_f$](0,2.2) to[vR=$R_x$] (1.5,2.2)  to[short](1.5,0);
\end{circuitikz}
\caption{Inverting variable amplifier}\label{Non-inverting variable amplifier}
\end{minipage}%
\begin{minipage}{.5\textwidth}
  \centering
\begin{circuitikz}
\draw(0,0) node[op amp] (opamp) {}
	%(opamp.+) node[left] {$v_+$}
	%(opamp.+) ++ (-.3,0) node[ground] {} -- (opamp.+) 
	(opamp.out) to[short] (1.8,0) node[right] {$v_o$}
	(opamp.down) ++(0,-.7) node[below] {$-v_{cc}$} -- (opamp.down)
	(opamp.up) ++ (0,.7) node[above] {$+v_{cc}$} -- (opamp.up)
	(opamp.down) ++ (0,-.25)to[C,/tikz/circuitikz/bipoles/length=1cm] (1,-.8)node[ground,rotate = 90,yshift = 1em] {}
	(opamp.up) ++ (0,.25)to[C,/tikz/circuitikz/bipoles/length=1cm] (1,.8)node[ground,rotate = 90,yshift = 1em] {};
	\draw(-4,-.5)node[left]{$v_1$} to[R=$R_{1}$,o-] (-2,-.5) to[short] (opamp.+);
	\draw(-4,-1.5)node[left]{$v_2$} to[R=$R_{2}$,o-] (-2,-1.5) to[short] (-2,-.5);

	\draw(opamp.-) -- (-1.5,.5) to[short](-1.5,2.2) to[R=$R_4$](1.5,2.2) to[short](1.5,0);
	\draw(-1.5,2.2) to[R=$R_3$] (-4,2.2)node[ground] {};
\end{circuitikz}
\caption{Non-inverting summing amplifier, unitary gain}
\end{minipage}
\end{figure}
\item Emitter follower test: at first we built a circuit without the emitter follower using an input impedence of $R=100.2\pm 1\, \text{k}\Omega$ and a load of $R_L=19.8\pm 2\,\text{k}\Omega$.  Then we added the op-amp stage and compared the output measurements in the 2 different cases.
\begin{figure}[H]
\centering
\begin{minipage}{.5\textwidth}
  \centering
\begin{circuitikz}
\draw(0,0)node[ground]{} to[sV] (0,2) to[R=$R$,-o]node[right,xshift=.8em] {$v_o$} (2,2);
\draw(1.8,2) to[R=$R_L$](1.8,0)node[ground]{};
\end{circuitikz}
\caption{Test circuit without follower}
\end{minipage}%
\begin{minipage}{.5\textwidth}
  \centering
\begin{circuitikz}
\draw(0,0) node[op amp] (opamp) {}
	%(opamp.+) node[left] {$v_+$}
	(opamp.+) ++ (-.3,0) node[ground] {} -- (opamp.+) 
	(opamp.out) to[short,-o] (1.8,0) node[right] {$v_o$}
	(opamp.down) ++(0,-.7) node[below] {$-v_{cc}$} -- (opamp.down)
	(opamp.up) ++ (0,.7) node[above] {$+v_{cc}$} -- (opamp.up)
	(opamp.down) ++ (0,-.25)to[C,/tikz/circuitikz/bipoles/length=1cm] (1,-.8)node[ground,rotate = 90,yshift = 1em] {}
	(opamp.up) ++ (0,.25)to[C,/tikz/circuitikz/bipoles/length=1cm] (1,.8)node[ground,rotate = 90,yshift = 1em] {};
	\draw(-4,-1) to[sV,l=$v_{in}$] (-4,.5) to[R=$R$] (-2,.5) to[short] (opamp.-);
	\draw(-4,-1) node[ground] {};
	\draw(-1.5,.5) to[short](-1.5,2.2)to[short] (1.5,2.2)  to[short](1.5,0);
	\draw(1.6,0) to[R=$R_L$] (1.6,-2)node[ground]{};
\end{circuitikz}
\caption{Test circuit with follower}
\end{minipage}
\end{figure}

\item Current generator: the aim of this circuit is to generate a stable fixed current indipendent from the load. We generated a $1$ mA current using a DC voltage source of 5 V and a 4.9693 $\pm$ 0.7 k$\Omega$ resistor. The load was simulated with a trimmer. 
\item Differential amplifier: the full equation the circuit in figure \eqref{differential amplifier} is the following:
\[v_o = \frac{R_F}{R_1}\left[\frac{v_b}{1+R_f/Ry}\left(1+\frac{R_1}{R_f}\right)-v_a\right]\]
we first set to ground $v_b$, in this way we were able to set up the gain of the circuit (we chose it to be $A=2$ with $R_F =3\pm 0.2\, \text{k}\Omega$ (5\% error of nominalvalue)). After that we put the same signal of $v_a$ in $v_b$ with a resistor $R_f$ and a variable resistor $R_y$ made with $R_2$ in series with a trimmer. We managed with the trimmer to get the output as close to zero as possibile. This means in the equation $R_f/R_y = R_1/R_F$ so the new output is exaclty what we want $v_0 = A(v_b-v_a)$. For testing the amplifier we used $v_a= 5$ V DC and for $v_b$ a sine wave 1 V peak-peak 100 Hz with an offset of 5 V.
\end{itemize}

\begin{figure}[H]
\centering
\begin{minipage}{.5\textwidth}
  \centering
\begin{circuitikz}
\draw(0,0) node[op amp] (opamp) {}
	%(opamp.+) node[left] {$v_+$}
	(opamp.+) ++ (-.3,0) node[ground] {} -- (opamp.+) 
	(opamp.out) to[short] (1.8,0) node[right] {$v_o$}
	(opamp.down) ++(0,-.7) node[below] {$-v_{cc}$} -- (opamp.down)
	(opamp.up) ++ (0,.7) node[above] {$+v_{cc}$} -- (opamp.up)
	(opamp.down) ++ (0,-.25)to[C,/tikz/circuitikz/bipoles/length=1cm] (1,-.8)node[ground,rotate = 90,yshift = 1em] {}
	(opamp.up) ++ (0,.25)to[C,/tikz/circuitikz/bipoles/length=1cm] (1,.8)node[ground,rotate = 90,yshift = 1em] {};
	\draw(-4,-.8) to[battery1] (-4,.5) to[R=$R_{3}$] (-2,.5) to[short] (opamp.-);
	\draw(-4,-.5) node[ground] {};
	
	\draw(-1.5,.5) to[short](-1.5,2.2) to[vR=$R_x$] (1.7,2.2)to[myvoltmeter](1.7,0);
\end{circuitikz}
\caption{Current source generator}
\end{minipage}%
\begin{minipage}{.5\textwidth}
  \centering
\begin{circuitikz}
\draw(0,0) node[op amp] (opamp) {}
	%(opamp.+) node[left] {$v_+$}
	%(opamp.+) ++ (-.3,0) node[ground] {} -- (opamp.+) 
	(opamp.out) to[short] (1.8,0) node[right] {$v_o$}
	(opamp.down) ++(0,-.7) node[below] {$-v_{cc}$} -- (opamp.down)
	(opamp.up) ++ (0,.7) node[above] {$+v_{cc}$} -- (opamp.up)
	(opamp.down) ++ (0,-.25)to[C,/tikz/circuitikz/bipoles/length=1cm] (1,-.8)node[ground,rotate = 90,yshift = 1em] {}
	(opamp.up) ++ (0,.25)to[C,/tikz/circuitikz/bipoles/length=1cm] (1,.8)node[ground,rotate = 90,yshift = 1em] {};
	\draw(-4,-.5)node[left]{$v_B$} to[R=$R_{f}$,o-] (-2,-.5) to[short] (opamp.+);
	\draw((-2,-.5) to[vR=$R_{y}$] (-2,-2.5) node[ground]{};

	\draw(opamp.-) -- (-1.5,.5) to[short](-1.5,2.2) to[R=$R_F$](1.5,2.2) to[short](1.5,0);
	\draw(-1.5,2.2) to[R=$R_1$,-o] (-4,2.2)node[left] {$v_A$};
\end{circuitikz}
\caption{differential amplifier}\label{differential amplifier}
\end{minipage}
\end{figure}
\section{Data analysis}
\begin{figure}[H]
\centering
\includegraphics[width=.7\textwidth]{2/Variable_amplifier.png}
\caption{Variable amplifier}\label{Variableamplifier}
\end{figure}
In the inverting amplifier we used a trimmer in order to vary the gain, in fact the equation is:
\[v_{o} = -v_{in}\frac{R_f+R_x}{R_{in}}\]
so increasing $R_x$ cause output to linear increasing. The output voltage is limited by the op-amps's power supply voltage, it cannot increase further and the signal goes flat, as we can see in figure \eqref{Variableamplifier} (light blue line), this behavior is called "Clipping". The graphic also shows a discrepance between the absolut value of maximum and minimum voltage during the clipping: this is due to the asimmetry between \emph{pnp} and \emph{npn} transistors in the op-amp's final stage.\\

In the non-inverting summing amplifier circuit we wanted the output to be the simple sum of the signals in entrance, that were identical (sines, 1V peak-peak, 100Hz). We used
