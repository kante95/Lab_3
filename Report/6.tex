\chapter{Building an electronic thermometer}
We can manage to measure a temperature analysing the voltage drop between the ends of a resistor in which flows a known curent, having the resistance depending on temperature by a fixed formula.
We build this electronic thermometer, testing during the process the new coponents.

\section{Materials}
\begin{itemize}
\item Operational amplifiers OP07
\item Instrumentation amplifier (INA) AD622
\item Precision +5V Voltage Reference REF02
\item thermistor PT100
\item Resistors, trimmers
\item Power supply RIGOL DP831A
\item Waveform generator RIGOL DG1032
\item Multimeter RIGOL DM3068
\end{itemize}
MANCANO VALORI ED ERRORI 

\section{Experimental setup}

\begin{figure}[H]
\centering
\begin{circuitikz}
%ponte
\draw(0,0);
\draw(0,6.7)--(0,7.2);
\node[above] at (0,7.2) {$v_{+}$};
\draw(-1,6.7)--(1,6.7);
\draw(-1,6.7)--(-1,6.2);
\draw(1,6.7)--(1,6.2);
\draw (-1,6.2) to[R,l^=$R_1$](-1 ,4.7);
\draw (1,6.2) to[R,l^=$R_2$](1,4.7);
\draw(-1,4.7)--(-1,3.5);
\draw(1,4.7)--(1,3.5);
\draw (-1,3.5) to[R,l^=$R_4$](-1 ,2);
\draw (1,3.5) to[vR,l^=$R_3$](1,2);
\draw(-1,2)--(-1,1.5);
\draw(1,2)--(1,1.5);
\draw(-1,1.5)--(1,1.5);
\node[sground] at (0,1.5) {};
%opamp
\draw(-1,4.69)--(4,4.69);
\draw(1,3.71)--(4,3.71);
\node[op amp] (opamp) at (5,4.2) {};
\draw(5,5.3)--(5,4.7);
\node[above] at (5,5.3) {$+v_{cc}$};
\draw(5,3.0)--(5,3.7);
\node[below] at (5,3.0) {$-v_{cc}$};
\node[right] at (6.19,4.2) {$v_{out}$};

\end{circuitikz}
\caption{Ponte}\label{Ponte}
\end{figure}

