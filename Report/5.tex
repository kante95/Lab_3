\chapter{A new device: the comparator}
We can use an op-amp to compare two input signals: its output corresponds to a logic value (let's say 0) if $v_- > v_+$, to the other (1) in the opposite case. Verified this feature for two different op-amps, we build a Schmitt's trigger in order to adjust the comparator sensitivity and we try it in a twilight switch circuit.

\section{Materials}
\begin{itemize}
\item Operational amplifier UA741
\item Operational amplifier LM311
\item Fototransistor OP550A
\item Resistors, trimmer, LED, capacitors
\item Power supply RIGOL DP831A
\item Waveform generator RIGOL DG1032
\item Multimeter RIGOL DM3068
\item Oscilloscope RIGOL MS02102A
\end{itemize}
MANCANO VALORI ED ERRORI DI CONDENSATORE, RESISTORI,LED

\section{Experimental setup}
We have always powered the op-amps with a $\pm15$ V DC voltage and, in order to reduce possible noises, we added two capacitors to both op-amps pins for the power supply and the ground, one of 100nF and the other of 10nF (4 in total).\\
At first we used the UA741 as a comparator in order to build a relaxation oscillator producing a square wave from a capacitor charge and discharge: we chose $R_1 = R_2 = 10k\Omega$ and a 0.1 nF capacitor. The circuit (see figure ?) has been tested with 5 different values of R: 1, 5, 10, 15, 20 k\Omega in order to have different periods and no imput signal was needed. A measure has been taken also setting the oscilloscope in single mode and then switching on the power supply.
We than tested the LM311 as both a non-invertent and an invertent comparator using $R_L = 1k\Omega$ and a triangular input signal of ? V and ? Hz.\\
Regarding the Schmitt's trigger, we added to the previous circuit the resistences $R_1 = 10 k\Omega$ and $R_2 = 100 \Omega$ in order to shift the reference voltage of a factor $\frac{1}{100}$ upward and downward and analyzed the behaviour at the point when $v_{in}\approx v_{ref}$.\\
At last, we built the twilight switch:\\
DA FINIRE E RIVEDERE

