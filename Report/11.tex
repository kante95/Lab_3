\chapter{System stability analysis and Logic circuits}
In the first  part of the experience we implemented the cicuits latch SR, latch SR synchronized (flip flop) and FF type D, by using just NAND and NOT gates. Later we built an anti-bounce latch SR and a circuit that registers an impulsive input by storing it in the memory. Later we used an J-K FF shift register.

\section{Materials}
\begin{itemize}
\item Operational amplifiers OP07
\item Instrumentation amplifier (INA) AD622
\item Precision +5V Voltage Reference REF02
\item Resistors, trimmers
\item Power supply RIGOL DP831A
\item Waveform generator RIGOL DG1032
\item Multimeter RIGOL DM3068
\item NAND 74LS00
\item NOT 74LS04
\item 74LS109 
\end{itemize}
The resistors used were all with an uncertainty of 5\%

\section{Experimental setup}
We built the latch SR as in the figure and for visualizing the output we  connected $Q$ and $\overbar{Q}$ to an 8-led chip. Later we modified ur circuit as in figure and we verified that when EN is low voltage the circuit is in the HOLD configuration, that is it keeps the memory of the previous state. We modified again the circuit to be an FF type D (see fig.) and we verified that every time the EN is on the bit on D is registred in the memory.\\
Then we  built a latch SR with an ``anti-bounce'' configuration, in particular we made sure the output was changing from LOW to HIGH without too much noise.\\
We later built an on/off system that had a HIGH output on when we activated SET for a short amount of time and a LOW one when we did the same with REST.\\
We also built a circuit that gave as output a square wave with frequency that was half or a quarter of  the clock, depending on where we took the output. This was done by using two J/K FF connecting the clock with the output of the previous  FF or the clock (for the first FF) and connecing J to $V_{cc}$ and K to ground.\\
We built a shift register with four J/K FF used as FF type. We connected the FF in circle connecting the output of one with the input of the other. We also used the pin CL and PR to preset the bit in the registers with the help of a capacitor that kept the voltage LOW for a short amount of time. We visualizied the shift register connecting the output of each FF to some LED.\\
Laslty we built a counter with two 74LS191 an a D-FF. The FF was  used to set if we wanted to count up or down by connecting the output to the D / $\overline{U}$. We connected the clock to the elements of the cicuit for  setting the frequency of each bit. We used the same method used in the shift register for presetting the bits in the counter. For using an 8 bit system we needed to activate the second 74LS191 just when the bits of the first were all on, this was done by connecting RCD to G of the second one used to enable.
\section{Data analysis}
We can see in the plots for the anti-bounce system that the input has a lot more noise than the output, this it's exaclty what we expected.\\
The plot X shows the various frequencies of the frequency divider.\\
