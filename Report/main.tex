\documentclass[oneside]{book}

\usepackage{mathtools}
\usepackage{graphicx}
\usepackage[utf8]{inputenc}
\usepackage{float}
\usepackage{tabularx}
\usepackage[toc,page]{appendix}
\usepackage{gensymb}
\usepackage{subcaption}
\usepackage{tikz}
\usepackage{circuitikz}
\usepackage{array}
\usepackage{booktabs}
\usepackage{colortbl}
\usepackage{xcolor}
\usepackage{xfrac}
\usepackage[a4paper, inner=1.5cm, outer=3cm, top=3cm, 
bottom=3cm, bindingoffset=1cm]{geometry} 

\usepackage{fancyhdr}
\renewcommand{\chaptername}{Experiment}


\begin{document}
\begin{titlepage}

\newcommand{\HRule}{\rule{\linewidth}{0.5mm}} % Defines a new command for the horizontal lines, change thickness here

\center % Center everything on the page
 
%----------------------------------------------------------------------------------------
%	HEADING SECTIONS
%----------------------------------------------------------------------------------------
\includegraphics[width = 50mm]{unitn.jpg}\\[0.5cm]
\textsc{\LARGE Università degli studi di Trento}\\[1cm] % Name of your university/college
\textsc{\Large Group \textbf{MAR01}}\\[0.5cm] % Major heading such as course name

%----------------------------------------------------------------------------------------
%	TITLE SECTION
%----------------------------------------------------------------------------------------

\HRule \\[0.4cm]
{ \huge \bfseries REPORT OF THE EXPERIMENTS\\ PERFORMED IN THE COURSE OF\\[0.3cm] PHYSICS LABORATORY III}\\[0.2cm] % Title of your document
\HRule \\[1.5cm]
 
%----------------------------------------------------------------------------------------
%	AUTHOR SECTION
%----------------------------------------------------------------------------------------

\begin{minipage}{0.4\textwidth}
\begin{flushleft} \large
\emph{Authors:}\\
Canteri Marco\\Biasi Lorenzo\\Luca Vespucci % Your name
\end{flushleft}
\end{minipage}
~
\begin{minipage}{0.4\textwidth}
\begin{flushright} \large
\emph{Professor:} \\
Rolly Grisenti % Supervisor's Name
\end{flushright}
\end{minipage}\\[3cm]

% If you don't want a supervisor, uncomment the two lines below and remove the section above
%\Large \emph{Author:}\\
%John \textsc{Smith}\\[3cm] % Your name

%----------------------------------------------------------------------------------------
%	DATE SECTION
%----------------------------------------------------------------------------------------

{\large \today}\\[3cm] % Date, change the \today to a set date if you want to be precise

%----------------------------------------------------------------------------------------
%	LOGO SECTION
%----------------------------------------------------------------------------------------

%\includegraphics{Logo}\\[1cm] % Include a department/university logo - this will require the graphicx package
 
%----------------------------------------------------------------------------------------

\vfill % Fill the rest of the page with whitespace

\end{titlepage}

\tableofcontents
\chapter{Basic circuits with an operational amplifier}
In this experiment we have built five different circuits. The first is an open loop circuit with the operational amplifier uA741, the goal was to find the maximum voltage outputed by the op-amp as justified from the equation $v_o = A_{ol}(v_+-v_-)$ where $A_{ol}$ tends to infinity in the ideal model. The last four circuits are in closed loop configuration with a negative feedback, they consist on a follower, a non inverting amplifier, an inverting amplifier and a weighted summing amplifier. We have measured the voltage input and the voltage output of every circut.
\section{Materials}
\begin{itemize}
\item Operational amplifier uA741
\item Resistors, nominal value: 100 $\ohm$, 220 $\ohm$
\item Power supply RIGOL DP831A
\item Waveform generator RIGOL DG1032
\item Multimeter RIGOL DM3068
\item Oscilloscope AGILENT 54261A
\end{itemize}
\section{Experiment setup}
In the first four circuits the output of the waveform generator was a sine wave of 100Hz frequency and a peak-peak voltage of 100mV.
We measured the waveform output signal and the output voltage $v_o$ of the ap-amp. The measurements were performed using an oscilloscope triggered externally, the signal acquired is an 8 cyles average. The voltage supply of the op-amp was set to $v_{cc} = 15$V for all the circuits.\\
For the last circuit we used another sine wave signal with the same 100Hz frequency and a different peak-peak voltage. The oscilloscope's setting and the measurement taken was the same as before.
\begin{figure}
\centering
\begin{circuitikz}
 \draw(0,0) node[op amp,yscale=-1] (opamp) {}
%(opamp.+) node[left] {$v_+$}
(opamp.-) ++ (-.3,0) node[ground] {} -- (opamp.-) 
(opamp.out) node[right] {$v_o$}
(opamp.up) ++(0,-.5) node[below] {$-v_{cc}$} -- (opamp.up)
(opamp.down) ++ (0,.5) node[above] {$+v_{cc}$} -- (opamp.down);
\draw(-3,-1) to[sV] (-3,.5) to[short] (opamp.+);
\draw(-3,-1) node[ground] {};
\end{circuitikz}
\caption{Open loop circuit}
\end{figure}
\begin{figure}[H]
\centering
\begin{circuitikz}
 \draw(0,0) node[op amp,yscale=-1] (opamp) {}
%(opamp.+) node[left] {$v_+$}
(opamp.-) ++ (-.3,0) -- (opamp.-) 
(opamp.-) ++ (-.3,0) -- (-1.5,-1.8) -- (1,-1.8) -- (1,0)
(opamp.out) node[right] {$v_o$}
(opamp.up) ++(0,-.5) node[below] {$-v_{cc}$} -- (opamp.up)
(opamp.down) ++ (0,.5) node[above] {$+v_{cc}$} -- (opamp.down);
\draw(-3,-1) to[sV] (-3,.5) to[short] (opamp.+);
\draw(-3,-1) node[ground] {};
\end{circuitikz}
\caption{Follower}
\end{figure}
\begin{figure}[H]
\centering
\begin{circuitikz}
 \draw(0,0) node[op amp,yscale=-1] (opamp) {}
%(opamp.+) node[left] {$v_+$}
(opamp.-) ++ (-.3,0) -- (opamp.-) 
(opamp.-) ++ (-.3,0) to[R = $R_1$] (-3,-0.5) to (-3,-1) node[ground]{}
(opamp.-) ++ (-.3,0) -- (-1.5,-1.8) to[R,l_=$R_2$] (1,-1.8) -- (1,0)
(opamp.out) node[right] {$v_o$}
(opamp.up) ++(0,-.5) node[below] {$-v_{cc}$} -- (opamp.up)
(opamp.down) ++ (0,.5) node[above] {$+v_{cc}$} -- (opamp.down);
\draw(-4,-1) to[sV] (-4,.5) to[short] (opamp.+);
\draw(-4,-1) node[ground] {};
\end{circuitikz}
\caption{Non inverting amplifier}
\end{figure}
\begin{figure}[H]
\centering
\begin{circuitikz}
 \draw(0,0) node[op amp,yscale=-1] (opamp) {}
%(opamp.+) node[left] {$v_+$}
(opamp.-) ++ (-.3,0) -- (opamp.-) 
(opamp.-) ++ (-.3,0) to[R = $R_1$] (-3.4,-0.5) to[sV] (-3.4,-1.7) node[ground]{}
(opamp.-) ++ (-.3,0) -- (-1.5,-1.8) to[R,l_=$R_2$] (1,-1.8) -- (1,0)
(opamp.out) node[right] {$v_o$}
(opamp.up) ++(0,-.5) node[below] {$-v_{cc}$} -- (opamp.up)
(opamp.down) ++ (0,.5) node[above] {$+v_{cc}$} -- (opamp.down);
\draw(-4.3,.5)node[ground] {} to[short] (opamp.+);
\end{circuitikz}
\caption{Inverting amplifier}
\end{figure}
\section{Data analysis}
\end{document} 
